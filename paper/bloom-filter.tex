\section{Bloom filter usage}
\label{sec:bloom_filter}

    %TODO: reference
    A bloom filter is a probabilistic data structure which allows a fast check
    for a potential membership of an item in a set $B$.
    As a data structure, it consists of an array of a constant number $m$ of
    bits $b_0$ to $b_{m-1}$.
    For an empty set, all those bits are set to $0$.
    On insertion of an element into the set, $k$ hash functions $h_0$ to
    $h_{k-1}$ are applied to the element.
    The resulting values are transformed to positions within the array, e.g.
    through the modulo operation.
    Those bits are then set to $1$.
    Given a set of $n$ elements, the membership of an element $e$ may be checked
    by checking whether each bit ``pointed'' to by the hash functions applied to
    the new element are set.
    If one of those bits is not set, the element is not in the corresponding set
    $B$.

    \begin{equation}
        \exists_i : \neg b_{h_i(e)\%m} \Rightarrow e \notin B
    \end{equation}

    If all the bits relevant for the element are set, the element \emph{may} be
    in the set $B$:

    \begin{equation}
        \forall_i : b_{h_i(e)\%m} \Rightarrow
        p(e \in B) = \left(1 - \left(1 - \frac{1}{m} \right)^{kn} \right)^k
    \end{equation}

    In libreset, bloom filters will be used as filters to speed up some
    operations.
    The use of bloom filters will, however, exceed the classical use for
    speeding up lookups.
    The filter may also be used to speed up merges and intersections by allowing
    early recognition of disjunctive sets.


